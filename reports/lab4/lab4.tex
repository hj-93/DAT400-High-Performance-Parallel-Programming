\documentclass[a4paper, DIV12, headsepline]{scrartcl}

% common packages
\usepackage{lmodern}
\usepackage[T1]{fontenc}
\usepackage[utf8]{inputenc}
\usepackage[english]{babel}
\usepackage{amsfonts}
\usepackage{amssymb}
\usepackage{amsmath}
\usepackage{siunitx}
\usepackage{graphicx}
\usepackage{url}
\usepackage{listings}
\usepackage{tikz}
\usepackage{enumitem}
\usepackage[labelfont=bf]{caption}

% set head and foot
\usepackage{scrpage2}
\pagestyle{scrheadings}
\clearscrheadfoot
\ihead{Lab 4 -- Report}
\ohead{Group 25: Hui Jing (cid: huij), Tobias Fuchs (cid: fuchs)}
\cfoot{\pagemark}

% set pdf options
\usepackage[pdfborder={0 0 0}, bookmarksopen=true, bookmarksnumbered=true, pdftitle={Lab 4 Report}, pdfauthor={Hui Jing, Tobias Fuchs}, pdfsubject={Report}]{hyperref}

\begin{document}

\section*{Report for Lab 4}
\subsection*{Task 1 -- OpenMP parallel for}
\begin{verbatim}
#pragma omp parallel
{
    int thread_id = omp_get_thread_num();
    int tcount = omp_get_num_threads();

    std::cout << "Thread ID " << thread_id << std::endl;
    if (thread_id == 0) std::cout << "Total threads: " << tcount << std::endl;

    #pragma omp for
    for ( ... ) {
        ...
    }
}
\end{verbatim}
\begin{enumerate}
\item The thread id inside the parallel region is printed out with the aid of the \texttt{omp\_get\_thread\_num()} function.

\item The total number of threads inside the parallel region is printed out with the aid of the \verb|omp_get_num_threads()| function. It is printed only if \verb|thread_id == 0|. Therefore, it is output only once.
\end{enumerate}


\subsection*{Task 2 -- Parallel performance gains}
\begin{enumerate}
\item We use \texttt{time} command to time the total execution time of the program and use \texttt{omp\_get\_wtime()} to time the parallel execution. 
\begin{verbatim}
    double total_time_in_parallel = 0; // variable in global namespace
    double start = omp_get_wtime();
    ... parallel code ...
    double end = omp_get_wtime();
    total_time_in_parallel += end - start;
\end{verbatim}
Regardless of number of threads, the time spent on serial execution is nearly fixed: about 2.55s. The parallel execution time decreases when increasing number of threads, Table blow shows the results.

%Parallel Execution: 64.6382seconds
%real    1m7.191s
%user    1m7.125s
%sys     0m0.052s
%
%Parallel Execution: 33.3797seconds
%real    0m35.900s
%user    1m10.450s
%sys     0m0.048s
%
%Parallel Execution: 17.8018seconds
%real    0m20.382s
%user    1m17.310s
%sys     0m0.060s

\item The table below shows the execution time and efficiency:
\begin{table}[htbp]
\centering
\sisetup{table-number-alignment=left,table-format=2.4,table-auto-round}
\begin{tabular}{rSSSS}
\hline
No of Threads: & {1} & {2} & {4} \\
\hline
Serial: & \SI{2.5528}{s} & \SI{2.5203}{s} & \SI{2.5802}{s}  \\
Parallel: & \SI{64.6382}{s} & \SI{33.3797}{s} & \SI{17.8018}{s}  \\
Efficiency: & {1} & {0.968226} & {0.907748}  \\
\hline
\end{tabular}
\caption{Efficiency for different number of threads}
\label{tab:tab22}
\end{table}
\end{enumerate}

% Serial Impl:
% Total time in dot: 10.9268s
% Total time (all): 13.4743s
% Percentage in dot: 81.0938%
% Efficiency = 10.5384s / (1 * 13.4743s) = 0,7821
% 
% 1 Processor:
% Total time in dot: 8.0824s
% Total time (all): 10.5384s
% Percentage in dot: 76.6945%
% Efficiency = 10.5384s / (1 * 10.5384s) = 1
% 
% 2 Processors:
% Total time in dot: 4.16418s
% Total time (all): 6.67957s
% Percentage in dot: 62.342%
% Efficiency = 10.5384s / (2 * 6.67957s) = 0,7889
% 
% 4 Processors:
% Total time in dot: 2.11744s
% Total time (all): 4.64428s
% Percentage in dot: 45.5925%
% Efficiency = 10.5384s / (4 * 4.64428s) = 0,5673


\subsection*{Task 3 -- OpenMP scheduling policies}
\begin{table}[htbp]
\centering
\sisetup{table-number-alignment=left,table-format=2.4,table-auto-round}
\begin{tabular}{rSSSS}
\hline
 & {(no param)} & {1} & {16} & {128} \\
\hline
schedule(static, <param>) & \SI{4.6989}{s} & \SI{7.6719}{s} & \SI{4.7186}{s} & \SI{6.2623}{s} \\
schedule(dynamic, <param>) & \SI{10.9646}{s} & \SI{10.8904}{s} & \SI{4.8008}{s} & \SI{6.3657}{s} \\
schedule(guided, <param>) & \SI{4.8178}{s} & \SI{4.8473}{s} & \SI{4.7503}{s} & \SI{6.3849}{s} \\
\hline
\end{tabular}
\caption{Total running times for all loop scheduling strategies.}
\label{tab:tab1}
\end{table}
\begin{enumerate}
\item We have included all three scheduling strategies, namely \verb|static|, \verb|dynamic|, and \verb|guided|, in our experiments.

\item Table~\ref{tab:tab1} shows the running times of all three strategies with either no parameter specified, a chunk size of 1, 16, or 128.

\item The optimal policy is \verb|schedule(static)|. On the one hand,  since all loop iterations of the \textsc{Gemm} loops take more or less the same time, there is no benefit of using \verb|dynamic| scheduling because the extra time spend for scheduling does not improve the distribution of work significantly. On the other hand, \verb|schedule(static)| is more cache-locality preserving than the \verb|guided| scheduling strategy wherefore the \verb|static| policy wins the comparison.
\end{enumerate}


\subsection*{Task 4 -- Which loop to parallelize?}
All task 4 is done using 4 threads.
\begin{enumerate}
\item  j loop(column) parallelism implementation:
\begin{verbatim}
    for( int row = 0; row < m1_rows; ++row ) {
       #pragma omp parallel for shared(row) //task 4.1
        for( int col = 0; col < m2_columns; ++col ) {
            for( int k = 0; k < m1_columns; ++k ) {
                output[ row * m2_columns + col ] += m1[ row * m1_columns + k ] * m2[ k * m2_columns + col ];
             }
\end{verbatim}

\item For k loop(entry) parallelism implementation,  each output vector element will be written simultaneously by all threads, therefore we decided to use reduction on a local variable and copy the value to each vector element  when parallel execution is done:
\begin{verbatim}
     for( int row = 0; row < m1_rows; ++row ) {
         for( int col = 0; col < m2_columns; ++col ) {
             float sum = 0;
             #pragma omp parallel for shared(row, col) reduction(+: sum)
             for( int k = 0; k < m1_columns; ++k ) {
                 sum += m1[ row * m1_columns + k ] * m2[ k * m2_columns + col ];
             }
            output[ row * m2_columns + col ] = sum; }}
\end{verbatim}
\item Table below shows the parallel execution time of the 3 strategies.

Apparently, parallelizing  the k loop is the worst strategy because the critical region basically eliminates the benefits of multithreading, and even worse, the thread synchronization added extra cost.

Parallelizing i loop(row)  slightly outperforms parallelizing j loop(column), which can be explained by cache false sharing. Compared with i loop parallelism, j loop parallelism means output vector accesses by each thread is more interleaved, i.e. higher chance of false sharing.
\begin{table}[htbp]
\centering
\sisetup{table-number-alignment=left,table-format=2.4,table-auto-round}
\begin{tabular}{rSSSS}
\hline
Parallelized loop: & {i} & {j} & {k} \\
\hline
Parallel execution time: & \SI{17.8018}{s} & \SI{19.7789}{s} & \SI{186.638}{s}  \\
\hline
\end{tabular}
\caption{performance for different strategies}
\label{tab:tab43}
\end{table}
%Parallel Execution: 17.8018seconds
%Parallel Execution: 19.7789seconds
%Parallel Execution: 186.638seconds
\end{enumerate}

% 4)
% 4.1)
% #pragma omp parallel
% {
%     for( int row = 0; row < m1_rows; ++row ) {
%         for( int k = 0; k < m1_columns; ++k ) {
%             #pragma omp for
%             for( int col = 0; col < m2_columns; ++col ) {
%                 ...
%             }
%         }
%     }
% }
% 
% Total time in dot: 146.123s
% Total time (all): 148.722s
% Percentage in dot: 98.2529%
% 
% 4.2)
% #pragma omp parallel
% {
%     for( int row = 0; row < m1_rows; ++row ) {
%         #pragma omp for
%         for( int k = 0; k < m1_columns; ++k ) {
%             for( int col = 0; col < m2_columns; ++col ) {
%                 ...
%             }
%         }
%     }
% }
% 
% Total time in dot: 26.3541s
% Total time (all): 28.9587s
% Percentage in dot: 91.006%
% 
% 4.3)
% - Overhead for each #pragma omp for
% - In Task 3, only invoked once per call of the function
% - In Task 4.1, #pragma omp for called m1_rows * m1_columns times per function call
% - In Task 4.2, #pragma omp for called m1_rows times per function call
% 


\subsection*{Task 5 -- Exploring SIMD}
\begin{enumerate}
\item Without any loop permutations, there is not any vectorizing happening for any of the loops in the \textsc{Gemm} function.

\item With the loop permutation from Lab~2, the optimization flags \verb|-O1| and \verb|-O2| yield also no vectorization of the loops. However, using the optimization flag \verb|-O3| vectorizes the inner-most loop (column loop; \texttt{src/vector\_ops.cpp:242:39: note: loop vectorized}). The second inner-most loop ($k$ loop), however, is not vectorized (\texttt{src/vector\_ops.cpp:241:31: note: bad data references.}). Also, the outer-most loop (row loop) is not vectorized either (\texttt{src/vector\_ops.cpp:240:31: note: not vectorized: multiple nested loops.}).

\item With no vectorized \textsc{Gemm} loops (optimization flag \verb|-O2|), the program has the following running time.
\begin{verbatim}
Total time in dot: 30.82s
Total time (all): 33.5521s
Percentage in dot: 91.8574%
\end{verbatim}
With the vectorized inner-most loop (optimization flag \verb|-O3|), the program's running time improves as seen below.
\begin{verbatim}
Total time in dot: 10.9205s
Total time (all): 13.4652s
Percentage in dot: 81.1018%
\end{verbatim}
The total speedup is $S = \frac{\SI{33.5521}{s}}{\SI{13.4652}{s}} = 2.4918$.
\end{enumerate}


\end{document}
